\section{Ejercicio 3}

Para resolver este ejercicio se modicara \texttt{screen.c}, \texttt{screen.h}, \texttt{mmu.c}, \texttt{mmu.h}, \texttt{kernel.asm} y \texttt{defines.h}.

\subsection{a}
Para comenzar vamos a modificar \texttt{screen.c} y \texttt{screen.h}, creamos la funcion en \texttt{C} \texttt{screen\_refrescar} que se encargara de limpiar la pantalla y dibujar el fondo usando \texttt{screen\_inicializar} y luego escribir la informacion del sistema (Los datos de las tareas de ambos jugadores y los puntajes) usando la funcion \texttt{print} (Brindada por la catedra).

\subsection{b}
Definimos \texttt{DIR\_PAGINAS\_KERNEL} como \texttt{0x27000} en \texttt{defines.h}, luego vamos a \texttt{mmu.h} y creamos \texttt{mmu\_inicializar\_dir\_kernel}, esta funcion crea en la posicion de memoria de \texttt{DIR\_PAGINAS\_KERNEL} el directorio de tablas de paginas y en la siguiente pagina (\texttt{0x28000}) la primera tabla de paginas que posee las paginas de \texttt{0x0} a \texttt{0x3FFFFF} con identity mapping.

\subsection{c}
Vamos a \texttt{kernel.asm} y llamamos el extern \texttt{mmu\_inicializar\_dir\_kernel} para crear el directorio de tablas y las tablas. Luego muevo a \textbf{CR3} la posicion del directorio de tablas de paginas (\texttt{DIR\_PAGINAS\_KERNEL}). Luego para activar paginacion seteamos el bit \textbf{PG} (Paging) de \textbf{CR0}.

\subsection{d}
Para terminar en \texttt{screen.c/h} creamos la funcion \texttt{print\_group} que usa print para escribir el nombre del grupo (Alineado a la derecha) y la llamamos desde \texttt{kernel.asm}, con esto podemos confirmar que la pagina esta funcionando.