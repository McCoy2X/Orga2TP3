\section{Ejercicio 4}

Para resolver este ejercicio se modicara \texttt{mmu.c}, \texttt{mmu.h} y \texttt{kernel.asm}.

\subsection{a}
Comenzamos editando \texttt{mmu.c} y creando la funcion \texttt{mmu\_inicializar} que se asigna la primera pagina de la memoria libre para guardar valores del sistema. Y en \texttt{0x100000} guarda un puntero a la siguiente pagina util.

\subsection{b}
Ahora escribimos \texttt{mmu\_inicializar\_dir\_pirata} que se encargara de crear el directorio de pagina de las tareas, que pide dos paginas libres y crea el directorio de tablas en la primera y una tabla de paginas en la segunda, haciendo identity mapping de los primeros 500MB. Y luego mapeamos un sector de la memoria que no esta asignado a nada (\texttt{0x401000}) a la posicion del mapa correspondiente (\texttt{X:1,Y:1} si es el primer jugado y \texttt{X:78,Y:48} para el segundo jugador), copia el codigo de la tarea (Que se pasa como parametro a la funcion) para terminar desmapeamos la posicion mapeada anteriormente.

\subsection{c}
Para la tarea anterior es neceario crear las dos funcciones \texttt{mmu\_mapear\_pagina} y \texttt{mmu\_unmapear\_pagina} que con un \textbf{CR3}, una direccion virtual y una fisica en el caso de mapeo o una direccion virtual en el caso de desmapeo modifica el directorio de tablas y la tabla de paginas de la \textbf{CR3} pasada como parametro y mapea o desmapea la pagina.

\subsection{d}
Para este punto vamos a \texttt{kernel.asm} y creamos un segmento de codigo para testear las cosas echas. Llamamos a \texttt{mmu\_inicializar\_dir\_pirata} y cambiamos el \textbf{CR3} del sistema con el que devuelve la funcion. Modificamos la memoria de video para cambiar el color de fondo del primer caracter de la pantalla. Esto luego fue eliminado del kernel.asm como pide el subindice.